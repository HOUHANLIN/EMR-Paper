% 材料与方法 / 实验设计部分(占位数据使用 \TODO 标记)

\makeatletter
\@ifundefined{c@todoctr}{\newcounter{todoctr}}{}
\makeatother
\providecommand{\TODO}{%
  \stepcounter{todoctr}%
  {\small\textbf{\textit{\textsf{*\thetodoctr}}}}}

\section{实验设计(Study Design)}

本研究采用回顾性测试与前瞻性验证相结合的系统开发与评估方案,旨在构建并评估一个基于语音交互与大型语言模型的智能电子病历生成系统。核心创新在于将大语言模型限定为“信息提取与映射器”,采用模板化槽位填充范式,通过严格限定模型角色与输出形式提升生成内容的安全性与可控性。

研究整体遵循一个“迭代优化”的闭环验证框架(图~\ref{fig:overall_workflow}),该框架在总体上对系统进行统一评估,并通过回顾性测试与前瞻性临床队列验证相结合的双重测试体系逐层推进,既完成前期性能核验,又实现临床场景落地验证。整体流程依次包含以下五个阶段:

\begin{enumerate}
\item 数据输入与 ASR 预处理:将医生口述语音转换为规范化文本;
\item LLM 信息提取与结构化映射:从文本中精准提取关键信息并填入结构化模板槽位;
\item 信息验证与纠正:对提取信息进行自动化逻辑检查与纠错;
\item 病历生成与输出:生成结构化电子病历并进入人机协同审核流程;
\item 迭代优化:基于审核与评估反馈持续优化系统组件与参数。
\end{enumerate}

实验重点在于比较本系统与普通生成式大模型(不依赖模板填充的自由生成模式)在信息准确性、幻觉控制、结构规范性和书写效率等方面的差异。

\subsection{双重测试体系设计}

为实现“实验室性能校验$\rightarrow$临床场景验证”的全链路评估,本研究构建回顾性测试与前瞻性临床队列验证的双重测试体系,二者在设计上形成结构对称、目标互补的验证逻辑,为系统性能的全面评估提供支撑。

\subsubsection{回顾性测试}

\paragraph{测试目的}

承担系统临床准入前的基础性能核验职能,评估系统是否具备进入临床应用的核心能力,为后续前瞻性临床评估筑牢前置验证基础;同时验证系统在跨专科场景下的信息提取与病历生成适配性,初步判定其技术可行性。

\paragraph{测试数据来源}

从光华口腔医院信息系统中抽取经脱敏处理的真实电子病历构成回顾性测试集,病例专科分布以口腔种植专科为主,同时纳入口腔正畸、牙周病等相关科室病例,用于验证跨专科适配性;脱敏标准为去除姓名、联系方式、住址、证件号码等直接身份标识及可间接识别个体的诊疗关联隐私信息,脱敏流程经医院伦理委员会审核。

\paragraph{测试流程与评估指标}

对每份真实病例,由经过训练的志愿者根据原始病历内容在安静环境中完成语音复刻,获得与电子病历一一对应的语音文件。以脱敏电子病历作为金标准病历,围绕本系统设计两类测试任务:

\begin{itemize}
\item \textbf{基准比对}:以金标准病历为评估基准,从信息提取准确度、文本相似度、术语规范性、幻觉发生率等多维度对系统输出进行量化评估;
\item \textbf{效率对照}:同步统计传统键盘录入生成同等质量病历所需的时间与修改负担,形成系统辅助下与传统书写方式的基础效能对比数据。
\end{itemize}

\subsubsection{前瞻性临床队列验证}

\paragraph{研究对象与分组}

选取 202X 年 X 月—202X 年 X 月光华口腔医院种植科共 30 名临床医师作为研究对象,按随机数字表法分为试验组(15 名,使用智能电子病历生成系统)和对照组(15 名,使用传统键盘录入方式)。

\paragraph{纳入与排除标准}

纳入标准如下:

\begin{enumerate}
\item 光华口腔医院种植科执业医师,工作年限 $\geq 1$ 年;
\item 熟练掌握传统键盘录入病例的方法,具备基本计算机操作能力;
\item 自愿参与本研究,签署知情同意书;
\item 试验期间能保证至少 80\% 的工作时间使用指定录入方式。
\end{enumerate}

排除标准如下:

\begin{enumerate}
\item 存在语言表达障碍或听力障碍,无法正常使用语音输入功能;
\item 工作岗位为行政或科研岗,临床病例录入工作量每周 $<5$ 例;
\item 试验期间因休假、调岗等原因无法完成全程测试;
\item 对测试系统相关技术存在过敏或使用禁忌。
\end{enumerate}

\paragraph{样本量与试验周期}

参考既往类似研究,以病例录入耗时为主要结局指标,假设对照组平均录入耗时为 12 min/例,试验组为 6 min/例,标准差均为 3 min,$\alpha=0.05$,$\beta=0.2$,采用配对 t 检验样本量公式计算,需每组至少 12 名医师;考虑 10\% 脱落率,最终纳入 30 名医师,每组 15 名。试验周期为 3 个月。

\paragraph{观测指标}

所有数据采用 SPSS 26.0 进行统计分析,核心观测指标分为三类:

\begin{itemize}
\item 效率指标:两组医师单份病例录入总耗时、日均完成病历数量;
\item 质量指标:病例完整性评分(参照金标准病历槽位覆盖标准)、信息准确率、诊疗逻辑一致性达标率;
\item 主观与安全指标:用户操作满意度评分(5 级 Likert 量表)、不良事件(数据泄露、系统故障)发生情况。
\end{itemize}

\subsection{研究对象与数据来源(Subjects and Data Sources)}

\subsubsection{研究对象与虚拟病例构建}

本研究采用“虚拟病例开发 + 真实病例测试”的设计。模型开发阶段(训练集与验证集)采用由口腔种植专科临床专家基于日常诊疗经验模拟构建的虚拟病例;模型测试阶段则使用来自真实临床电子病历的脱敏病例。

用于训练集与验证集的虚拟病例由来自光华口腔医院的 7 名口腔种植专科医生参与病例设计,最终构建了 420 个虚拟病人。每个虚拟病人均包含完整、连贯的病程与诊疗过程信息,至少包括:

\begin{itemize}
\item 人口学信息(虚构的年龄、性别等);
\item 主诉与现病史;
\item 既往史、过敏史、个人史及家族史;
\item 相关检查结果(如影像学与实验室检查,采用合理的模拟数值与描述);
\item 初步诊断与治疗计划;
\item 手术或随访记录(如适用)。
\end{itemize}

\subsubsection{语音与文本数据构建}

对于训练集与验证集中的每个虚拟病人,病例设计专家首先撰写一份结构化、规范化的黄金标准电子病历(golden EMR),作为后续系统输出对照文本。随后,专家在安静环境中,按照日常门诊问诊与病历书写习惯,以自然口语形式口述该虚拟病例的病情与诊疗过程,录制形成对应的语音文件。每个虚拟病例至少包含:

\begin{itemize}
\item 1 份专家撰写的黄金标准 EMR;
\item 1 段与之对应的专家口述语音(可覆盖不同就诊时间点或不同专家的描述);
\item 一份由研究团队整理的结构化病例要点,用于支持模板设计与字段定义完善。
\end{itemize}

对于回顾性测试集中的真实病例,除完成前述脱敏与语音复刻外,同步由研究团队基于原始病历形成对应的金标准 EMR;随后由经过统一培训的志愿者在安静环境中,参考原始电子病历内容,按照医生日常问诊与病历书写习惯以自然口语形式进行口述录音,获得与每份真实病历对应的语音文件。真实病例同样以口腔种植专科为主,并包含少量其他相关科室的病例。前瞻性试验的临床实时病历,则由参与医师在诊疗结束后,同步完成金标准病历的回溯标注,确保评估基准的统一性。

如语音录制中存在明显杂音、严重中断或长时间无效静音,要求重新录制,以保证语音数据质量满足 ASR 评估与系统验证需求。图 3 展示了系统界面的实际交互过程,从 *,到系统 *的关键步骤。

\subsubsection{伦理与合规性}

本研究涉及专家构建的虚拟病例、回顾性脱敏真实病历以及前瞻性临床试验数据。所有真实病例在纳入前均已完成去标识化处理,仅保留与研究相关的诊疗信息,不包含任何可直接或间接识别患者身份的个人隐私字段。前瞻性试验已通过医院伦理委员会审批(审批编号:\TODO),所有参与医师均签署知情同意书。

研究方案已提交 \TODO 伦理委员会/审查机构审查,并获得批准(审批编号:\TODO)。伦理委员会认定,本研究属于使用去标识化临床数据的低风险/豁免研究,允许在严格数据安全管理与访问控制前提下,将上述数据用于本课题的方法学开发与性能评估。

\subsection{数据预处理与标准化(Data Preprocessing and Normalization)}

\subsubsection{语音预处理与 ASR 模型配置}

所有原始语音经统一的预处理流程,包括:

\begin{itemize}
\item 环境噪声抑制与滤波;
\item 静音段检测与切分,去除长时间静默与无效片段;
\item 音频格式统一至采样率 \TODO Hz、位深 \TODO bit 的单声道 PCM 格式(或其他格式:\TODO)。
\end{itemize}

预处理后的语音输入至专科优化的自动语音识别(ASR)模型进行转写,生成初始文本。该 ASR 模型通过引入专科热词库提升对口腔种植相关术语、药品名称和检查项目的识别准确率。

ASR 模型信息如下:

\begin{itemize}
\item ASR 模型名称:\TODO;
\item 模型版本:\TODO;
\item 模型部署位置:\TODO(如院内 GPU 服务器 / 私有云 / 公有云服务等);
\item 训练/微调语料规模:约 \TODO 小时口腔专科语音数据;
\item 特征与声学参数:采样率 \TODO Hz,使用 \TODO 特征(如 MFCC / FBank 等);
\item 内部验证集上的性能:词错误率(Word Error Rate, WER):\TODO;句错误率(Sentence Error Rate, SER):\TODO。
\end{itemize}

上述指标用于表征 ASR 模块的基础性能,为后续系统整体性能分析提供背景。

\subsubsection{文本标准化处理}

对 ASR 转写文本与黄金标准 EMR 文本统一进行标准化处理,以降低非规范表达对评估与下游处理的影响,主要包括:

\begin{itemize}
\item 术语库(terminology lexicon)映射:将口语化表达、别名、缩写等映射为标准医学术语,如将 “松牙” 标准化为 “牙齿松动(\TODO)”,将通俗药品名映射为规范通用名或编码;
\item 错词库(error lexicon)纠错:基于以往 ASR 错误模式和本研究预实验中发现的常见错识别构建错词库,对高频错误进行自动化替换与纠正。
\end{itemize}

经上述处理,最终获得两类规范化文本对:(1)规范化转写文本;(2)规范化黄金标准 EMR,用于后续信息提取评估与文本相似度计算。

\subsubsection{数据集划分}

以病例为单位,采用“虚拟病例用于开发、真实病例用于测试”的划分策略:将全部虚拟病例随机划分为训练集与验证集,比例为 \TODO : \TODO;经脱敏处理的真实病例单独构成独立测试集。

\subsection{系统流程与模型方法(System Workflow and Model Methods)}

系统流程包括 ASR 识别与文本标准化、模板匹配与槽位设计、LLM 信息处理以及病历生成与人工审核。实验设置两种核心模式进行对比:模板填充 + 信息提取模式(实验组)与普通生成式(非模板填充)模式(对照组)。在两种模式中,ASR 模型、LLM 型号与推理参数、医生工作流与交互界面、评估口径与统计方法均保持完全一致,\textbf{唯一自变量}为生成环节是否引入“槽位模板约束”的结构化抽取范式。整体端到端临床落地流程如图~\ref{fig:system_workflow} 所示,从语音输入及 ASR 预处理开始,经 LLM 信息提取与槽位填充,最终生成结构化电子病历,并在 \TODO 等场景中形成闭环应用。

\subsubsection{模板库设计与管理}

\paragraph{模板来源与构建依据}

模板库由口腔种植专科医生基于临床路径、院内病历书写规范及相关诊疗指南协同设计而成。主要依据包括:

\begin{itemize}
\item 院内标准化电子病历模板文档:\TODO;
\item 国内外口腔种植相关诊疗指南与专家共识:\TODO;
\item 专家对日常临床文书书写习惯的经验总结。
\end{itemize}

模板库覆盖口腔种植常见场景,如初诊病历、复诊及随访、术前评估和术后/并发症随访等。目前模板库包含 \TODO 份结构化模板,由 \TODO 名专科医生共同审阅确认。

\paragraph{槽位结构与字段定义}

每份模板由若干预定义槽位(slots)组成,用于承载结构化信息,典型槽位包括但不限于:基本信息、主诉、现病史、既往史/过敏史/个人史/家族史、口内/体格检查要点、辅助检查结果、初步诊断与鉴别诊断、治疗计划、手术记录与术中要点、随访与疗效评价。

为便于系统实现与后续统计分析,每个槽位定义以下属性:槽位名称与说明;数据类型(文本型、枚举型—值域为 \TODO、日期型—统一采用 \TODO 格式、数值型—单位为 \TODO);必填/选填标记;是否允许多值(如药品清单、既往史等)。具体字段列表与数据类型配置在本研究中以附录或补充材料形式给出(记为 \TODO)。

\paragraph{模板更新机制}

为适应临床实践变化和指南更新,模板库设计了版本管理与更新机制。触发条件包括:新版诊疗指南或规范发布(如 \TODO 年版 \TODO 指南)、临床专家反馈的槽位缺失或定义不合理、试用阶段评估显示模板影响信息完整性或可用性。

模板更新流程为:至少 \TODO 名专科专家联合提出修改建议;召开模板评审会议形成修订版本;为模板分配新版本号(如 v\TODO),旧版本归档保存;在系统中统一切换至新版本模板。本研究所有虚拟病例均基于模板库的 \TODO 版本生成,以保证实验一致性。

\subsubsection{LLM 模型配置与任务定义}

\paragraph{LLM 模型信息}

为保证方法透明与可复现性,本研究明确给出用于信息提取和生成对照的 LLM 配置信息:

\begin{itemize}
\item 模型名称:\TODO;
\item 模型版本:\TODO;
\item 参数量:约 \TODO 亿参数(\TODO B);
\item 部署位置:\TODO(如院内 GPU 服务器 / 私有云集群 / 公有云 API 等);
\item 推理环境与框架:如 PyTorch \TODO + CUDA \TODO / 其他:\TODO;
\item 最大上下文长度:\TODO tokens;
\item 解码策略:temperature = \TODO,top-k = \TODO,top-p = \TODO。
\end{itemize}

上述模型配置在实验组与对照组中保持一致,确保对比公平。

\paragraph{实验组:信息提取 + 槽位映射模式}

在模板填充系统中,LLM 被限定为“信息提取器”,任务是从规范化转写文本中提取与各槽位对应的原始文本片段,而非生成新内容。

\textbf{输入}:规范化转写文本;对应病历模板的槽位定义与字段说明。

\textbf{任务定义}:解析每个槽位所需的信息类型与格式要求;在转写文本中定位与该槽位匹配的原始表述;将结果以结构化形式输出为“槽位 $\rightarrow$ 文本片段”的映射表。

\textbf{输出形式}:可为 JSON 或其他结构化格式(具体为 \TODO),其中每个字段对应一个模板槽位及其抽取内容。

在该模式下,LLM 明确被指令要求不得发明或扩展转写文本中不存在的信息,从设计上降低医学幻觉风险。

\paragraph{对照组:普通生成式(非模板填充)模式}

对照系统使用同一 LLM 模型,但采用自由生成范式:输入为规范化转写文本,提示词例如“请根据以下问诊记录撰写一份完整的口腔种植电子病历”,不提供模板与槽位信息;模型直接生成结构化或半结构化病历文本,不受槽位结构约束;生成文本直接进入评估与医生修改环节,用于模拟“ASR + 通用生成式 LLM”的常见应用方式。

\subsubsection{病历生成与人机协同审核}

实验组中,系统生成的“槽位 $\rightarrow$ 文本片段”映射结果被自动填入模板,生成结构化电子病历草案。草案呈现给临床专家或模拟使用者,进行逐项审核与必要修改,形成“系统辅助生成版病历”。对照组中,医生直接对普通生成式 LLM 输出的完整病历文本进行审核与修改。

\subsection{金标准病历的界定与信息类核心评测标准}

为确保双重测试体系下评估结果的客观性与专业性,本研究首先明确金标准病历的核心定义,并建立针对信息覆盖率与信息提取度的量化评测标准,为后续多维度评估提供统一基准。

\subsubsection{金标准病历的核心定义}

本研究中的\textbf{金标准病历}是指基于口腔种植专科临床诊疗规范、院内电子病历书写标准及权威诊疗指南构建的,具备完整结构、规范术语、准确诊疗逻辑的结构化电子病历,是评估智能病历生成系统性能的基准参照。其定义包含以下三层核心属性:

\begin{enumerate}
\item \textbf{结构完整性}:需严格匹配口腔种植专科预设病历模板的全量槽位,涵盖基础信息模块(患者基本人口学特征、就诊时间/科室)、诊疗信息模块(主诉、现病史、既往史/过敏史、口腔专科检查、辅助检查结果)、诊疗决策模块(初步诊断/鉴别诊断、治疗计划、手术记录/随访建议)共 3 大类 18 个核心槽位(具体槽位清单见附录),无关键信息模块缺失。
\item \textbf{术语规范性}:所有医学表述需符合《口腔种植学名词》《国际口腔医学术语标准》及院内专科术语库要求,禁止口语化或非规范缩写(如“种牙”需规范为“口腔种植修复术”,“松牙”需规范为“牙体松动(分度需明确)”);药品、器械名称需匹配国家药监局标准名录及院内耗材编码,检查项目需对应影像学/实验室标准命名。
\item \textbf{诊疗逻辑一致性}:病历中各模块信息需具备临床诊疗逻辑闭环,如“主诉–检查结果–诊断结论–治疗方案”需因果对应(例:主诉“上颌前牙缺失 3 月”需对应“上颌 11、12 牙列缺损”的检查结论,及“上颌前牙区种植体植入 + 冠修复”的治疗计划),且无诊疗方案与患者病史(如过敏史)冲突的逻辑谬误。
\end{enumerate}

\subsubsection{金标准病历的来源与质控体系}

本研究结合不同测试场景的特性,为金标准病历设定差异化的来源路径,并配套建立针对性的质控体系,以保障评估基准的统一性与可靠性:

\begin{enumerate}
\item \textbf{虚拟病例金标准}:由具有 $\geq$ \TODO 年口腔种植专科临床经验的执业医师,依据院内规范病历模板及典型诊疗场景完成虚拟病例的初版撰写,直接作为模型开发与验证阶段的金标准病历,病例构建过程严格遵循专科诊疗共识,确保病历内容契合临床实际诊疗逻辑与文书规范。
\item \textbf{回顾性测试集金标准}:从光华口腔医院信息系统临床归档病历中直接抽调口腔种植专科医师完成的真实电子病历,此类病历为临床诊疗结束后按院内文书规范完成的归档版本,已纳入医院常规临床文书质控流程,可直接作为回顾性测试的评估基准。
\item \textbf{前瞻性验证金标准}:由参与试验的临床医师在完成当日诊疗工作后,同步对照院内种植科规范病历模板,结合实时诊疗记录完成标准病历的撰写,作为前瞻性临床队列验证的评估基准。
\item \textbf{统一质控抽检机制}:对上述三类来源的金标准病历按 10\% 比例实施分层随机抽检,由科室质控专家对抽检病历的槽位完整性、术语规范性及诊疗逻辑关联性进行复核,通过抽检结果实现对批次金标准病历整体质量的核验与把控。
\end{enumerate}

\subsubsection{信息覆盖率的评测标准}

信息覆盖率用于评估智能系统生成病历对金标准病历中核心信息的完整覆盖程度,分为字段级覆盖率与内容级覆盖率两个维度。

\paragraph{字段级覆盖率}

字段级覆盖率定义为系统生成病历中已填充的必填槽位数量占金标准病历总必填槽位数量的比例,聚焦“结构层面的信息完整性”。计算公式为:
\[
\text{字段级覆盖率} = \frac{\text{系统生成病历已填充的必填槽位数量}}{\text{金标准病历总必填槽位数量}} \times 100\%.
\]
字段级覆盖率 $\geq 95\%$ 视为“优秀”,85\%--94\% 为“良好”,75\%--84\% 为“合格”,$<75\%$ 为“不合格”;其中“初步诊断”“治疗计划”等核心诊疗槽位若未覆盖,直接判定为“不合格”。

\paragraph{内容级覆盖率}

内容级覆盖率定义为系统生成病历中各槽位内实际包含的有效诊疗信息点数量,占金标准病历对应槽位信息点总量的比例,聚焦“内容层面的信息完整性”。单槽位与整体内容覆盖率分别定义为:
\[
\text{单槽位内容覆盖率} = \frac{\text{系统槽位中匹配金标准的有效信息点数量}}{\text{金标准对应槽位信息点总量}} \times 100\%,
\]
\[
\text{整体内容覆盖率} = \frac{\sum (\text{单槽位内容覆盖率} \times \text{槽位临床权重})}{\sum \text{槽位临床权重}} \times 100\%.
\]
权重根据槽位临床重要性设定,如“初步诊断”权重 0.2,“既往史”权重 0.08,具体权重表见附录。以口腔种植专科为例,“口腔专科检查”槽位的信息点需细化至牙位(如 11、26)、牙槽骨骨量(高度/宽度)、邻牙状态等最小诊疗单元;“治疗计划”需细化至种植体品牌/型号、植骨方式、修复体类型等具体信息。

\subsubsection{信息提取度的评测标准}

信息提取度用于评估智能系统从语音输入中提取关键信息并映射至金标准病历槽位的精准度,结合客观量化指标与专科主观核验实现多维评估。

\paragraph{客观量化指标}

基于信息检索领域的经典指标,对各槽位信息分别计算精确率(Precision)、召回率(Recall)与 F1 分数:
\[
\text{精确率} = \frac{\text{系统提取的正确信息数量}}{\text{系统提取的总信息数量}} \times 100\%,
\]
\[
\text{召回率} = \frac{\text{系统提取的正确信息数量}}{\text{金标准病历中的总信息数量}} \times 100\%,
\]
\[
F_1 = 2 \times \frac{\text{精确率} \times \text{召回率}}{\text{精确率} + \text{召回率}}.
\]
针对口腔种植专科高价值信息(如种植体型号、牙槽骨骨量数值、手术入路、术后医嘱时长),单独计算提取准确率,要求此类信息提取精确率与召回率均需 $\geq 98\%$,方可判定为“专科适配性达标”。

\paragraph{专科主观核验标准}

邀请 3 名未参与系统开发的口腔种植医师,对系统提取信息的临床可用性进行 5 级量表评分(1 表示完全不可用,5 表示无需修改可直接使用),评分维度包括术语匹配度(提取术语是否符合专科规范)、数值准确性(骨量、植体型号等量化信息是否无偏差)和逻辑关联性(提取信息是否与患者整体诊疗场景匹配,无信息割裂或矛盾)。若客观 F1 分数 $\geq 0.9$ 但主观评分 $<4$ 分,则判定为“技术达标但临床适配不足”,需进一步优化术语映射规则。

\subsubsection{评测实施与质量控制}

评测主体由 1 名医学信息学工程师负责客观指标计算,2 名口腔种植专科医师完成主观核验。每份病例至少由 2 名医生在盲态下独立评分,评分者仅可见生成文本及对应的语音/黄金标准 EMR,不可见系统分组或生成策略信息;若任一维度的评分分歧 $>1$ 分,则引入第 3 名专家进行仲裁,并以仲裁后结果作为最终得分。

为评估主观评分的一致性,采用组内相关系数(Intraclass Correlation Coefficient, ICC)对 Likert 各维度分及总分进行一致性分析,并以 ICC $\geq 0.75$ 作为评分者间一致性“良好”的参考阈值;针对二分类指标(如是否存在幻觉、是否存在高风险错误等),采用 Kappa 检验评估评分者间一致性,要求 Kappa 值一般不低于 0.8,以确保评估结果的可靠性。

若语音录制本身存在信息缺失(如医生口述未提及过敏史),系统需在对应槽位标注“信息未获取”,此类情况不计入信息提取度误差,仅判定为“信息未覆盖”并纳入覆盖率统计。

\subsection{评估设计(Evaluation Design)}

研究从信息准确性、幻觉控制、文本质量与规范性以及书写效率四个维度对实验组与对照组进行比较。

\subsubsection{前瞻性生成策略对照试验设计}

在真实门诊场景中,为评估“槽位模板约束的结构化抽取范式”(实验组)与“非约束自由生成范式”(对照组)在临床上的差异,本研究在系统落地阶段预设前瞻性对照试验。医生始终在同一工作界面中完成病历书写,系统在后台根据随机化结果调用不同生成策略,医生对分组信息保持未知。

\paragraph{随机化方案 A:按就诊事件分配}

以“就诊事件”为单位进行 1:1 随机分配:同一位患者的一次门诊接诊(含完整问诊与记录过程)被随机分配至模板约束组或自由生成组。该方案在总体上可自然平衡两组的病种、病例复杂度与就诊时段分布。在实现上,对每次就诊的原始语音建立唯一的 ASR 文本 ID,统一由同一版本的 ASR 模型转写生成文本,两种生成策略只读取这一份转写结果,避免因 ASR 差异产生伪差异;在需要进行离线对照分析时,可在后台基于相同 ASR 文本同时生成两份草案,但前台仅向医生展示随机分配到的一组结果。

\paragraph{随机化方案 B:医生层面交叉设计}

考虑部分科室在排班与工作流上难以按就诊事件进行严格随机,本研究同时设计医生层面的交叉对照方案作为备选:以“医生/诊室”为单位分配生成策略,例如第 1 周某医生使用模板约束组,第 2 周切换为自由生成组(或顺序随机),以周或双周为固定切换周期。该方案通过同一医生在不同时间段内使用两种策略进行自我对照,有助于控制医生个体差异对结果的影响;切换周期需在“足够短以避免病例结构长期漂移”和“足够长以避免频繁切换造成工作混乱”之间取得平衡,实践中通常采用按周切换。

\paragraph{关键混杂因素与控制策略}

在前瞻性试验中,真实场景下最易失控的变量包括输入差异、界面差异导致的行为改变以及评估偏倚等。本研究通过以下策略进行控制:

\begin{itemize}
  \item \textbf{输入一致性控制}:针对同一次就诊,统一使用单一 ASR 模型生成的转写文本,并以唯一 ID 在系统内部标记;所有生成策略仅读取该文本。由此可避免“实验组 ASR 更好/对照组 ASR 更差”造成的伪差异。
  \item \textbf{界面与操作一致性}:两组均在完全一致的界面中呈现草案与字段面板,按钮、字段、提交方式及交互逻辑保持相同。对照组自由生成的文本在后台被映射到同一字段列表中展示(未能可靠映射时标记为“未提及”),以避免因结构化外观差异本身影响医生的修改行为和耗时。
  \item \textbf{评估盲法控制}:导出用于质量评估的病历文本时去除系统来源标识,评审医生仅基于文本内容及对应的语音/黄金标准 EMR 进行评分,无法辨别其来自哪一生成策略,从而降低评估偏倚。
  \item \textbf{模型与 ASR 版本锁定}:在整个实验周期内锁定 ASR 与 LLM 的模型版本与推理参数,并记录版本号;若因系统升级必须调整版本,则以分阶段分析方式处理,避免不同版本混合影响结果解释。
\end{itemize}

\subsubsection{信息提取与文本质量评估}

在测试集上,对两种系统分别评估:

\begin{itemize}
\item \textbf{信息提取准确度}:实验组直接将系统输出的每个槽位内容与对应病例的黄金标准 EMR 字段比对,计算精确率(Precision)、召回率(Recall)与 F1 分数;对照组由研究人员依据预定义规则从生成式文本中抽取相应字段,再与黄金标准 EMR 对照,计算相同指标。
\item \textbf{文本相似度与信息覆盖度}:使用 BERTScore 评估生成文本与黄金标准 EMR 在语义层面的相似性;使用 ROUGE-L 评估生成文本与黄金标准 EMR 的信息覆盖度和篇章重合度。
\end{itemize}

为便于系统性分析,本研究按照模板结构将病历内容划分为“一级槽位”和“二级字段”两个层级:一级槽位对应模板中的大段结构(如“基本信息”“主诉”“现病史”“辅助检查”“诊断”“治疗计划”等),用于宏观对比分布与结构完整性;在每个一级槽位下进一步拆分为若干二级字段,如“诊断”槽位下区分“初步诊断(枚举/文本)”“鉴别诊断(多值)”,“辅助检查”槽位下区分“影像结论(文本)”“关键数值(数值+单位)”,“基本信息”槽位下包含“年龄”“性别”“过敏史(枚举/多值)”等,用于细粒度的信息提取评估。

在字段级别,将系统输出与黄金标准 EMR 对应字段进行一一对照,采用如下 TP/FP/FN 判定规则:

\begin{itemize}
  \item \textbf{枚举型字段}:预测值在预先定义的允许值域内且与真值完全一致判为 TP;值域内但与真值不一致判为 FP;黄金标准存在而系统未输出或输出为空判为 FN。
  \item \textbf{数值型字段(含单位)}:在统一单位后,数值与单位均匹配(在预设容差范围内)判为 TP;数值或单位错误视为 FP;黄金标准存在而未输出判为 FN。
  \item \textbf{日期型字段}:先对日期格式进行标准化(如统一为 YYYY-MM-DD),标准化后与黄金标准完全一致判为 TP;无法映射或不一致视为不匹配,记为 FP 或 FN(取决于是否错误输出)。
  \item \textbf{文本型字段}:主分析采用严格或半严格匹配口径,即在去除标点、大小写归一化及常见同义词替换后,文本内容完全一致或高度重合视为 TP;明显不一致视为 FP;黄金标准存在但系统未给出对应内容视为 FN。对宽松匹配(如仅要求包含关键关键词)的结果仅用于敏感性分析。
  \item \textbf{多值字段}:如过敏史、并发症列表、药物清单等,按集合逐条统计 TP、FP、FN:预测与真值的交集部分计为 TP;仅在预测中出现的条目计为 FP;仅在真值中出现而预测缺失的条目计为 FN。
\end{itemize}

在此基础上,首先在字段级别汇总所有病例的 TP、FP、FN,计算 micro-averaged Precision、Recall 与 F1 作为结构化信息抽取性能的主要评估指标,并可按一级槽位维度进行分层统计,用于分析不同模块(如诊断、辅助检查、治疗计划等)的相对表现。

\subsubsection{幻觉与主观质量评估}

邀请 \TODO 名未参与系统开发与病例构建的口腔种植专科医生,采用盲评方式对测试集病历进行主观评价。每位专家随机阅读实验组和对照组生成的病历文本,并使用 5 级 Likert 量表从信息完整性、术语准确性、逻辑一致性、结构规范性、临床实用性、可读性、安全性等维度评分。评分锚点统一设定为“1 分 = 严重不足、3 分 = 基本可用、5 分 = 优秀”,并要求评分者仅基于对应病例的语音来源/黄金标准 EMR 判断,不受文本排版与界面风格影响。

各维度具体定义如下:

\begin{itemize}
  \item \textbf{信息完整性}(黄金标准 EMR 中关键事实覆盖情况):
  1 分:关键槽位多处缺失(例如 $\geq 30\%$ 关键事实缺漏,整体严重不足);
  2 分:缺漏较多(约 20--30\%),但部分关键信息已覆盖;
  3 分:存在约 10--20\% 缺漏,整体基本可用;
  4 分:仅有少量非关键缺漏(约 5--10\%),整体较为完整;
  5 分:关键事实几乎全部覆盖(缺漏 $<5\%$),可视为接近完全覆盖。
  \item \textbf{术语准确性}(医学术语、牙位标记、用药名、检查项目名称等是否准确):
  1 分:多处明显用错或概念混淆,术语错误会影响理解或决策;
  2 分:术语错误或不规范用语较多,但通过结合上下文仍可大致理解;
  3 分:存在少量术语不规范或偶发错误,但整体不影响理解;
  4 分:术语基本规范、仅偶见轻微不规范表述;
  5 分:术语规范且无明显错误,与院内/指南术语高度一致。
  \item \textbf{逻辑一致性}(病史$\rightarrow$检查$\rightarrow$诊断$\rightarrow$治疗计划是否自洽):
  1 分:存在会误导诊疗的关键矛盾或明显逻辑错误;
  2 分:逻辑问题较多,需大幅度调整才能用于临床决策;
  3 分:存在可通过小幅修改解决的轻微不一致,整体逻辑基本可接受;
  4 分:整体逻辑自洽,仅有极少不影响决策的轻微不连贯之处;
  5 分:逻辑完全连贯,无明显矛盾或不一致。
  \item \textbf{结构规范性}(是否符合模板/院内书写规范,条目归类是否正确):
  1 分:结构混乱,大量段落/字段错放,明显违背规范;
  2 分:结构问题较多,重要信息分布和归类明显不合理;
  3 分:结构大体符合规范,存在少量错放或冗余;
  4 分:结构清晰,仅偶见轻微不规范或多余信息;
  5 分:结构严谨、字段归类准确,与模板/规范高度一致。
  \item \textbf{临床实用性}(是否可作为草案直接使用及修改成本):
  1 分:基本不能用,需整体重写;
  2 分:需大范围修改或补写才能使用;
  3 分:可用但需中等程度修改(多处增删或重写句段);
  4 分:仅需少量局部修改即可使用;
  5 分:几乎可直接签署或仅需极轻微修订。
  \item \textbf{可读性}(表述清楚程度、语句通顺度、口语/重复情况):
  1 分:难以阅读,句式混乱或大量口语/重复;
  2 分:勉强可读,但语言不够流畅、冗余明显;
  3 分:整体可读,存在一定冗余或不够精炼;
  4 分:语言较为流畅,仅偶见冗余或轻微不通顺;
  5 分:表述清晰简洁,逻辑顺畅,几乎无冗余或口语化。
  \item \textbf{安全性}(是否出现可能导致临床风险的错误/误导性内容,与幻觉/不符事实高度相关):
  1 分:存在至少 1 处经核实的高风险错误或明显违背临床常识的内容;
  2 分:存在多处需要警惕的潜在风险点,但经仔细核对可纠正;
  3 分:仅有不影响关键诊疗决策的低风险瑕疵(如局部表述不精确);
  4 分:整体安全,仅偶见可接受的轻微瑕疵;
  5 分:未发现明显风险点,在当前信息前提下可视为安全。
\end{itemize}

为增强主观量表与客观指标之间的一致性,在评分培训阶段向专家提供了带有客观指标标注的示例病历,用于对 1--5 分进行定标参考。具体而言,信息完整性与术语准确性维度主要与结构化抽取的 micro-averaged F1 以及文本相似度指标(BERTScore、ROUGE-L)之间建立如下对应关系:

\begin{itemize}
  \item 1 分:该病例字段级 micro F1 通常 $<0.6$,且 BERTScore/ROUGE-L 明显低于同场景基准分布;
  \item 2 分:字段级 micro F1 约在 0.6--0.7 区间,文本相似度整体低于中位水平,但明显优于 1 分对应的案例;
  \item 3 分:字段级 micro F1 约在 0.7--0.85 区间,文本相似度处于中等水平,可视为“基本可用”;
  \item 4 分:字段级 micro F1 约在 0.85--0.9 区间,BERTScore/ROUGE-L 高于中位数,仅有少量缺漏或偏差;
  \item 5 分:字段级 micro F1 通常 $\geq 0.9$,且 BERTScore 与 ROUGE-L 均接近或优于整体高分段水平,可视为与黄金标准高度一致。
\end{itemize}

安全性维度则与幻觉标注和高风险错误检测结果相联系,其与客观指标的对应关系为:

\begin{itemize}
  \item 1 分:存在至少 1 处经客观评估标记的高风险幻觉或严重违背临床常识的错误;
  \item 2 分:存在多处需警惕的潜在风险点(如被自动检测为疑似幻觉或潜在高风险用药),但经仔细核对可纠正;
  \item 3 分:仅有被标记为低风险、且不影响关键诊疗决策的错误或瑕疵;
  \item 4 分:未见高风险幻觉,且仅偶见可接受的轻微瑕疵(例如表述不够严谨但事实正确);
  \item 5 分:在当前检测与人工复核下未被标记出幻觉或高风险错误,可视为安全。
\end{itemize}

分析阶段可进一步计算各维度 Likert 得分与 micro F1、BERTScore、ROUGE-L 以及幻觉率之间的相关性,用于验证主客观指标的一致性。

安全性维度则与幻觉标注和高风险错误检测结果相联系:5 分要求该病例未被客观评估标记出幻觉或高风险错误;3 分允许存在个别被标注为低风险、不影响关键诊疗决策的错误;1 分一般对应存在至少 1 处经核实的高风险幻觉或严重违背临床常识的错误。分析阶段可进一步计算各维度 Likert 得分与 micro F1、BERTScore、ROUGE-L 以及幻觉率之间的相关性,用于验证主客观指标的一致性。

同时要求专家标注每份病历是否存在“幻觉”现象(定义为生成文本中出现与病例源文本〔虚拟病例脚本或真实电子病历〕及黄金标准 EMR 不符,且未在语音输入中出现的虚构信息或明显违背临床常识的内容),统计两种系统的幻觉发生率及高风险错误率。

\subsubsection{书写效率与修改负担评估}

为评估不同系统对病历书写效率的影响,选取 \TODO 名医生在统一实验环境下,使用以下三种方式完成一组病例的病历书写任务:传统键盘录入;对照组(普通生成式 LLM);实验组(模板填充式系统)。对每份病历记录从“开始录入/口述”到“最终确认”的总耗时、从系统输出到最终确认的修改时间,以及修改操作数量(如编辑/增删字段次数,记为 \TODO),比较实验组与对照组在总耗时与修改负担上的差异。

\subsubsection{消融实验}

为分析不同生成范式的独立贡献,在测试集上设计以下消融版本:

\begin{itemize}
\item 版本 A(自由生成式 LLM):保留 ASR 与文本预处理,但将 LLM 模式改为自由生成,不使用模板与槽位结构,由模型直接生成完整病历文本。
\end{itemize}

将完整系统(模板约束的信息提取与槽位填充模式)与版本 A 在 F1 分数、BERTScore、ROUGE-L、幻觉发生率以及专家主观评分方面进行比较,以验证信息提取范式相对于自由生成范式的实际贡献。

\subsection{统计学分析(Statistical Analysis)}

所有统计分析采用 \TODO 软件(如 SPSS \TODO 版或 R \TODO 版本)完成。主要比较对象包括:实验组(模板约束结构化抽取)与对照组(自由生成)在效率、质量、准确性与安全性等指标上的差异,以及智能电子病历系统与传统键盘录入方式在书写效率和主观体验上的差异。

\subsubsection{数据类型与描述方式}

本研究输出的指标主要分为三类:

\begin{itemize}
  \item \textbf{连续型指标}:包括医生修改耗时(秒/分钟)、修改次数(可视为计数型指标)、文本相似度指标(如 BERTScore、ROUGE-L)、Likert 各维度分及总分(1--5 分)等。
  \item \textbf{二分类指标}:包括是否出现幻觉(是/否)、是否出现高风险错误(是/否)、某一槽位/字段是否正确(对/错)、是否达到预设的质量达标阈值等。
  \item \textbf{结构化抽取指标}:针对每个槽位字段与黄金标准对比所得的 TP/FP/FN 计数,以及由此计算的 precision、recall 与 F1(可按病例或字段聚合)。
\end{itemize}

对连续型指标,首先采用 Shapiro--Wilk 正态性检验判断分布形态:若近似正态,则以均值 $\pm$ 标准差表示;若明显偏态或存在离群值,则以中位数(四分位距)表示。二分类指标以频数和百分比描述;结构化抽取指标在病例或字段层面计算后视为连续变量,以均值 $\pm$ 标准差或中位数(四分位距)报告。

\subsubsection{组间比较与具体检验}

对于同一病例在两种系统下均产生输出的配对设计(如离线对照生成或同一次就诊在后台同时运行两套生成策略),连续型指标采用配对 t 检验(近似正态)或 Wilcoxon 符号秩检验(非正态)进行比较;二分类指标(如该病例是否出现幻觉)采用 McNemar 检验或其扩展方法进行配对二分类比较;以病例为单位计算的槽位级 TP/FP/FN 及由此得到的 precision/recall/F1 同样采用上述配对检验方式。

对于每个病例仅随机进入一组的非配对设计(如部分前瞻性门诊试验或医生层面交叉设计的不同时间段),连续型指标采用独立样本 t 检验(正态)或 Mann--Whitney U 检验(非正态)进行比较;二分类指标采用卡方检验,当样本量较小或期望频数过低时采用 Fisher 精确检验。Likert 量表评分在一般情况下视为有序分类变量,优先采用非参数检验(如 Mann--Whitney U 检验或 Kruskal--Wallis 检验);在需进行多因素调整时,可辅以有序 Logistic 回归或线性混合效应模型。

在比较实验组与对照组时,除依赖随机化本身在总体上平衡病种与病例复杂度外,还记录并在必要时在回归模型中纳入年龄、性别、科室、首次/复诊、就诊时长、语音噪声评分等协变量,以进一步控制残余混杂因素。

\subsubsection{效应量、置信区间与多重比较}

除报告 $p$ 值外,所有主要结果均同时报告效应量及其 95\% 置信区间,以反映差异大小及不确定性。连续型指标可根据情形给出 Cohen's $d$ 或中位数差及其 95\% 置信区间;结构化抽取指标(precision/recall/F1)以组间差值及 95\% 置信区间描述。对于二分类结果(如幻觉率、高风险错误率等),报告风险差(Risk Difference, RD)、相对风险(Relative Risk, RR)或优势比(Odds Ratio, OR)及其 95\% 置信区间。

显著性水平统一设定为双侧 $p < 0.05$。当同时进行多指标或多组比较时,采用 \TODO 方法(如 Bonferroni 校正、Benjamini--Hochberg 程序等)控制多重比较带来的第一类错误累积风险。
