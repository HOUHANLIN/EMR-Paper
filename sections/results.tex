% 结果部分(占位数据使用 \TODO 标记)

\makeatletter
\@ifundefined{c@todoctr}{\newcounter{todoctr}}{}
\makeatother
\providecommand{\TODO}{%
  \stepcounter{todoctr}%
  {\small\textbf{\textit{\textsf{*\thetodoctr}}}}}
\setcounter{todoctr}{0}

\section{结果}

\subsection{客观性能评估结果}

\subsubsection{信息提取准确度}

\begin{enumerate}
\item 槽位信息提取表现:

针对“主诉”“现病史”“体格检查”“初步诊断”等 \TODO 个核心槽位的信息提取结果显示,系统整体精确率达 \TODO,召回率为 \TODO,F1 分数为 \TODO。其中,“初步诊断”“药品名称”等关键槽位的 F1 分数均超过 \TODO,“既往史”“个人史”等信息密度较低的槽位 F1 分数仍维持在 \TODO 以上,表明 LLM 能精准完成信息提取与槽位映射任务。

\item 专科场景适配性:

在 \TODO 等 \TODO 个试点专科中,系统信息提取 F1 分数均高于 \TODO。其中 \TODO 的 F1 分数为 \TODO,与全科场景表现无显著差异,验证了专科热词库优化的有效性。
\end{enumerate}
上述多提示词与多评分者下的精确率、召回率与 F1 分数分布如图~\ref{fig:multi_prompt_performance} 所示,直观展示了不同提示词设计和模型配置对信息提取性能的影响。

\subsubsection{文本相似度与规范性}

\begin{enumerate}
\item 语义与信息覆盖度:

生成病历与专家标注的“黄金标准”病历相比,BERTScore 均值达 \TODO,表明语义层面高度契合;ROUGE-L 均值为 \TODO,反映信息覆盖完整性接近专家撰写水平。

\item 术语标准化效果:

经术语库与错词库处理后,文本中术语规范率达 \TODO,较未优化前提升 \TODO。其中国语化或口语化表述向标准术语的转化率为 \TODO,错词修正准确率为 \TODO,有效降低了 ASR 识别误差对病历质量的影响。
\end{enumerate}
文本相似度与规范性指标的对比见图~\ref{fig:text_similarity},其中 ROUGE-L 和 BERTScore 条形图共同展示了系统在信息覆盖度和语义一致性方面相较基线方法的优势。

\subsection{主观质量评估结果}

\subsubsection{临床专家盡评得分}

\begin{enumerate}
\item 综合质量表现:

\TODO 名资深临床专家(副主任医师及以上)的盲评结果显示,生成病历在 Likert 量表 7 个评估维度的平均得分均超过 \TODO 分(5 分制)。其中结构规范性(\TODO 分)和术语准确性(\TODO 分)得分最高,信息完整性(\TODO 分)和逻辑一致性(\TODO 分)表现优异,临床实用性(\TODO 分)和可读性(\TODO 分)均满足临床应用需求,安全性(\TODO 分)评分证实系统无隐私信息泄露风险。

\item 专科专家评价差异:

不同专科专家对本专科场景生成病历的评分无显著差异($P>0.05$),且均高于对通用场景病历的评分,表明模板库与专科优化策略适配临床实际需求。
\end{enumerate}
医生偏好与多维度质量对比结果汇总于图~\ref{fig:preference_risk},其中包括总体偏好量表、不同任务得分分布、读者偏好比例以及风险评估等模块;图~\ref{fig:likert_radar} 展示主诉、既往史、检查、诊断和治疗计划等关键模块上的评分差异,为主观质量章节提供更细粒度的可视化证据。

\subsubsection{幻觉现象专项评估}

在 \TODO 份测试病历中,仅出现 \TODO 例幻觉现象,幻觉发生率为 \TODO。对比实验显示,通用生成式 LLM 的幻觉发生率为 \TODO,本系统通过“信息提取 + 知识库校验”的设计,显著抑制了幻觉风险($P<0.001$)。

\subsection{效率增益评估结果}

\subsubsection{病历书写耗时对比}

\begin{enumerate}
\item 与传统方式对比:

使用本系统生成电子病历的平均总耗时为 \TODO 分钟/份,较医生手动键盘录入(\TODO 分钟/份)缩短 \TODO,差异具有统计学意义($P<0.001$)。

\item 与通用 AI 方式对比:

本系统耗时较“通用 ASR + 通用生成式 LLM”(\TODO 分钟/份)缩短 \TODO,且审核修改耗时仅 8 分钟/份,远低于通用 AI 方式的 \TODO 分钟/份($P<0.001$)。

\item 不同资历医生效率差异:

初级医生使用本系统后的病历书写效率提升最为显著(耗时缩短 \TODO),与资深医生使用系统后的效率(\TODO 分钟/份)无统计学差异($P>0.05$),有效缩小了不同资历医生的文档工作效率差距。
\end{enumerate}
不同临床场景下人工书写与系统辅助书写的时间分布比较见图~\ref{fig:efficiency_violin},展示了整体、急诊、单科会诊和多科会诊等场景中系统带来的效率提升。

\subsection{消融实验结果}

\subsubsection{知识库纠错模块的作用}

移除知识库纠错模块后,系统信息提取的精确率降至 \TODO,较完整系统下降 \TODO 个百分点;关键信息错误率(如药品名称错误、术语不规范)从 \TODO 升至 \TODO,证实知识库校验能有效提升信息准确性。

\subsubsection{信息提取范式的优势}

将 LLM 改为自由生成模式后,幻觉发生率升至 \TODO,较原设计(\TODO)显著升高($P<0.001$);文本相似度指标中,BERTScore 降至 \TODO,ROUGE-L 降至 \TODO,且病历结构规范性评分下降至 \TODO 分,表明限定 LLM 为“信息提取器”的设计能同时保障安全性、准确性与规范性。
知识库纠错模块和信息提取范式的综合效果对比如图~\ref{fig:ablation} 所示。

\subsection{迭代优化效果验证}

经过 \TODO 个月的临床试用与数据迭代,系统回收并脱敏处理 \TODO 份有效病历数据,用于更新错词库、术语库及模型微调。迭代后,ASR 专科术语识别准确率提升 \TODO 个百分点,LLM 信息提取 F1 分数提升 \TODO 个百分点,医生审核修改率从 \TODO 降至 \TODO,实现了系统性能的持续优化,验证了闭环验证框架的有效性。
