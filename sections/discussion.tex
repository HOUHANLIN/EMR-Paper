\section{讨论}

本研究旨在设计并评估一款基于语音输入的生成式病历系统,该系统采用专科词库预处理、LLM 信息提取映射、召回校验等多重环节,有效辅助医生生成专业化、结构化病历。实验结果显示,该系统在生成病例的完整性、准确性、专业度方面基本符合标准病历要求(\textbf{部分存在改进空间}),在时效性上显著优于人工记录,系统生成病历的综合质量得到专家盲评认可。

本研究的结果与已有的 LLM 医学文本处理研究呈现出部分一致性,LLM 在处理医学文本时具有较高的完整性和准确性,效率显著优于人工记录\cite{song2025llm}。在处理常见的幻觉问题时,本系统也同样采用模型训练与再次核验相结合的设计\cite{zhang2025mr}。

特别值得关注的是,本系统采用信息提取与映射的思路,从医生语音中提取关键信息,填入模板的空白槽位,避免 LLM 模型无中生有,从源头上规避幻觉。虽已有研究关注分段化、模块化信息提取,但主要关注点在于提取信息的效率\cite{moser2025pipeline}。本研究采用提取模块化信息并填充模板,更关注消除 AI 幻觉现象。根据幻觉评估显示,本系统相对于通用 LLM 模型,幻觉出现率降低。

同时,模板化病历的标准化和完整性也得到验证。本系统以口腔种植科为例,提供多细分科室、多阶段临床模板,模板涵盖对应手术病历的关键信息。应用于其他临床学科时,可根据专科病历要求自由设计新模板,以适应不同科室的记录需求。

预处理和召回纠正环节在确保信息准确度上具有重要意义。预处理环节实现了对语音转录结果的初步专业化处理,召回纠正把控病历输出前的最后一关。两个环节均主要依托 LLM 模型。本系统的优势在于模型经过专业热词库、错词库、术语库的训练,能够显著提升 ASR 预处理和二次召回纠正的准确度与专业性。通过词库限定信息来源,可在一定程度上保证内容的可信度。随着实践应用中数据库的扩大与更新,词库数据将不断积累与演化,使模型能够灵活适应临床用词的发展变化,完成自我更新迭代。

\subsection{局限性}

本研究以口腔种植科为例,训练数据主要来源于单一医院(光华口腔医院),在样本规模、来源与学科覆盖范围方面存在局限性,可能导致模型存在一定偏差。

本研究成果与真实临床环境的适配度仍缺乏系统的实践检验。不同医生的病历书写习惯能否良好适配预设病历模板,罕见病和特殊情况是否能够通过系统进行充分而专业的处理,特殊信息(如牙位、扭矩等数字信息)能否被精确提取,以及隐私保护和医患双方接受度等实际应用问题,仍有待在后续研究中进一步探讨与验证。

\section{结论}

本研究构建的专科适配型语音生成电子病历系统,通过“ASR 专科词库优化 + LLM 槽位信息提取”的技术路径,有效降低了大模型幻觉风险,提升了病历生成的准确性与规范性;同时,本研究建立的回顾性与前瞻性双重测试体系,及金标准病历的界定与信息评测标准,完善了专科智能病历系统的评估体系。临床验证证实,该系统能显著提升临床病例录入效率和完整性,且保持较高的信息准确性,用户接受度良好,为口腔医疗场景下的病历数字化提供了实用且可控的解决方案,也为其他专科智能病历系统的研发与评估提供了参考范式。
