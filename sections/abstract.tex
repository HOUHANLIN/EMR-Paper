\section{介绍}

电子病历(EMR)作为临床诊疗过程的核心载体,其质量与生成效率直接关系到医疗安全与医护人员的工作负荷。然而,电子病历的普及在带来数字化便利的同时,也显著增加了医生的文档工作。研究表明,医生花费在病历书写上的时间可达其总工作时间的 30\% 以上,这种日益繁重的文档压力直接导致了职业倦怠与工作满意度下降。因此,探索能够高效、准确生成电子病历的新方法,已成为提升临床工作效率的关键课题。

语音输入技术被视为理想的解决方案之一,但现行方法在专科临床应用中存在明显瓶颈。首先,基于深度学习的通用自动语音识别(ASR)模型虽在通用领域表现良好,但对人名、地名及复杂的医学术语等低频词汇识别准确率不足,易出现误识别。其次,尽管大型语言模型(LLM)展现出强大的语言理解与生成能力,但其在生成完整病历文本时存在的“幻觉”现象,即输出错误或虚构信息,这给临床治疗带来了不可忽视的安全风险。

幸运的是,技术发展也为解决这些问题提供了契机。一方面,ASR 模型可通过引入专科热词库、错词库与术语库进行定向优化,显著提升对专业词汇的识别准确率;另一方面,基于 Transformer 架构的 LLM 具备强大的信息提取与映射能力,通过微调技术可使其精准地从文本中抽取关键信息片段,而非不可控地生成全文,这为从根源上降低“幻觉”风险提供了可能。已有研究证实,经过适配的 LLM 在临床文本汇总任务中可超越医学专家,这预示着其在减轻临床文档负担方面的巨大潜力。

目前,尤其缺乏针对口腔学科等专业场景的生成式病历软件。为此,本研究旨在创新性地将优化后的 ASR 模型与可控的 LLM 信息提取技术相结合,提出一种专科适配的语音生成电子病历新方法。该方法的核心在于通过词库优化 ASR 识别、引导 LLM 进行精准的槽位信息提取与绑定,从而在保证病历真实性与安全性的前提下,最大限度提升临床文档效率,为改善医护人员工作体验提供有效的技术支撑。
